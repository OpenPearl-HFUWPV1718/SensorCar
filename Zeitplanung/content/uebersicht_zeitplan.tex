\chapter{Projektübersicht und Planung}
\section{Ziel}
	Ziel des Projekts ist es, ein Modellauto zu bauen und mithilfe von OpenPEARL so zu programmieren, dass das Auto sich auf einer markierten Strecke orientieren, autonom fahren und auf Farbmarkierungen auf der Strecke reagieren kann.\\
	Der Aufbau des Autos greift auf fischertechnik zurück; die Orientierung soll mittels Lichtsensor, Gyroskop und Farbsensoren realisiert werden, das Fahren mit Schrittmotoren.\\
	Die Steuerung erfolgt von einem Raspberry Pi 3 aus mit OpenPEARL.
	
	\section{Teilprojekte}
	Das Projekt wird in folgende Phasen bzw. Teilprojekte unterteilt:
	
	\begin{itemize}
		\item \textbf{Projektstart}\\
		Einführung in das Projekt und Festlegung der Organisation. Grundlegende Abstimmung über Ziel und Umfang.
		\item \textbf{Einführung OpenPEARL und Philosophenproblem}\\
		 Zur Einarbeitung in die Programmiersprache OpenPEARL implementiert das Projektteam jeweils das Philosophenproblem.
		\item \textbf{Einrichtung der Raspberry Pis}\\
		 Für die weitere Arbeit am Projekt werden zwei Rasperberry Pi 3 eingerichtet, um mit OpenPEARL und NFS zu arbeiten. Der Zugriff auf die Rechner soll mittels \emph{ssh} möglich sein.
		\item \textbf{Inbetriebnahme und Test der Hardwarekomponente}\\  Die einzelnen Komponenten (Sensoren und Aktoren) werden als Teilprojekte separat am Raspberry Pi in Betrieb genommen und getestet.
		\item \textbf{Detailplanung des Gesamtsystem}\\  Die genaue Funktionalität und der Hardwareaufbau des Gesamtsystems werden festgelegt und dokumentiert.
		\item \textbf{Zusammensetzung Gesamtsystems}\\
		 Nachdem alle Komponenten erfolgreich in Betrieb genommmen und getestet wurden, wird das Gesamtsystem im Sinne des Ziels zusammengebaut. Daraufhin wird die Funktionalität des Gesamtsystems programmiert. Im Sinne eines inkrementell iterativen Vorgehens erfolgen Anforderungsanalyse, Implementierung und Tests bis das System die Anforderungen erfüllt.
		 \item \textbf{Projektabschluss}\\
		 Zum Abschluss des Projekts erfolgt die Fertigstellung der Dokumentation und Abnahme des Ergebnisses.
		 \item \textbf{Projektvorstellung}\\
		 Das Ergebnis des Projektes wird im Rahmen der Projektpräsentationen vorgestellt.
	\end{itemize}

	\section{Zeitplanung}
	Die untenstehende Tabelle detailliert die grobe Zeitplanung für den Projektablauf. Änderungen sind vorbehalten.\\
	
		\begin{tabular}{|p{5cm}|l|}
			\hline
			\textbf{Zeitraum} & \textbf{Teilprojekt / Aufgabe}\\
			\hline
			09.10.17 --- 16.10.17 & Projektstart\\
			\hline
			16.10.17 --- 23.10.17 & Einführung in OpenPEARL und Philosophenproblem\\
			\hline
			23.10.17 --- 30.10.17 & Einrichtung der Raspberry Pis\\
			\hline
			30.10.17 --- 27.11.17 & Inbetriebnahme und Test der Hardwarekomponenten\\
			\hline
			30.10.17 --- 27.11.17 & Detailplanung des Gesamtsystems\\
			\hline
			27.11.17 --- 15.01.18 & Zusammensetzung des Gesamtsystems\\
			\hline 
			15.01.18 --- 22.01.18 & Projektabschluss\\
			\hline
			26.01.18 & Projektpräsentation\\
			\hline
		\end{tabular}
	