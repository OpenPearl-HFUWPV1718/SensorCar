\chapter{Einleitung}

	Das Semesterprojekt OpenPEARL im Wintersemester 2017/18 beschäftigt sich mit der Anwendung von OpenPEARL in einem Beispielprojekt. 
	
	\section{OpenPEARL}
	PEARL steht für \emph{Process and Experiment Automation Realtime Language}, eine höhere Programmiersprache, die speziell dafür entworfen wurde, Multitasking-Aufgaben bei der Steuerung technischer Prozesse zu steuern. Die Sprache wurde um 1975 am IRT Institut der Leibniz Universität in Hannover entwickelt und 1998 vom Deutschen Institut für Normung in der DIN66253-2 standardisiert.\\
	OpenPEARL ist eine quelloffene Implementierung eines Kompilers und einer Laufzeitumgebung für PEARL. OpenPEARL befindet sich aktuell noch in der Entwicklung, ist aber weit genug fortgeschritten, um erste Beispielprojekte damit umzusetzen. In dieser Dokumentation werden deshalb auch in OpenPEARL gefundene Fehler festgehalten.
	
	\section{SensorCar}
	Ziel des Projektes ist es, ein autonomes Modellauto zu bauen, das, mit verschiedenen Sensoren ausgestattet, einer Linie folgen und auf Abruf weitere Manöver ausführen kann. Zur Umsetzung wird ein Raspberry Pi 3 mit mehreren Sensoren und Aktoren verbunden auf einem eigens konstruierten Fahrzeug aufgebracht und mit OpenPEARL programmiert. 
	
	
	