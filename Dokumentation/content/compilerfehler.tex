\chapter{Compilerfehler}
Hauptziel des Projekts ist das Testen von OpenPEARL in Anwendung. Tabelle  \ref{tab:fehler} gibt eine Übersicht über die gefundenen Fehler im OpenPEARL Compiler.

	\begin{table}[H]
	
	\begin{center}
		\renewcommand{\arraystretch}{1.2}
		\begin{tabular}{|p{0.2\textwidth}|p{0.6\textwidth}|p{0.2\textwidth}|}
			\hline
			\textbf{Fehler} & \textbf{Beschreibung} & \textbf{Kontext}\\
			\hline
			Belegung Semaphoren & Die atomare Belegung mehrere Semaphoren ist nicht möglich & Philosophen-problem \\
			\hline
			INIT BIT(4) & Führende 0 wird bei init von BIT-Arrays nicht berücksichtigt & Schrittmotor \\
			\hline
			TRY & TRY Anweisung gibt falschen Datentyp zurück & Gesamtsystem / Synchronisation \\
			\hline
			RET & Return ohne Argumente ist nicht möglich & Gesamtsystem / Menü \\
			\hline
			
		\end{tabular}
	\end{center}
	\caption{Gefundene Fehler im OpenPEARL Compiler}\label{tab:fehler}
\end{table}