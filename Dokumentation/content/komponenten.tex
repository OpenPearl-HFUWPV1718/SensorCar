\chapter{Komponenten}
\label{komponenten}

Im Folgenden sind die Konzepte und Herangehensweisen für die Programmierung der einzelnen Komponentenmodule beschrieben. Komplette, dokumentierte Module können im Ordner \emph{Code/Komponenten} des \href{https://github.com/OpenPearl-HFUWPV1718/SensorCar}{Repositories} eingesehen werden.

\section{Raspberry Pi 3}
Als Entwicklungs- und Zielplattform wird ein Raspberry Pi 3 genutzt, auf dem OpenPEARL installiert ist. Die Sensoren und Aktoren sollen wie in Tabelle \ref{tab:anschlussplan} am Raspberry Pi angeschlossen werden:

\begin{table} [H]
	\begin{center}
		 \renewcommand{\arraystretch}{1.5}
		\begin{tabular}{|l|l|l|}
			\hline
			\textbf{Gerät} & \textbf{BCM} & \textbf{Versorgung}\\
			\hline
			Lichtrechen & 6 -- 13 & 3,5 / 5V\\
			\hline
			Motortreiber& 16 -- 19, 20 -- 23 & extern\\
			\hline
			Farbsensor & 2, 3 (I$^2$C) & 3,3 / 5V\\
			\hline
			LEDs & 24 -- & 5V\\
			\hline
		\end{tabular}
	\end{center}
\caption{Anschlussplan Raspberry Pi}\label{tab:anschlussplan}
\end{table}

Der zugrundeliegende Belegungsplan ist im Anhang \ref{Belegungsplan} zu finden.

\section{LEDs}
Die LEDs können über einen Vorwiderstand direkt über jeweils einen GPIO-Pin des Raspberry Pi angesteuert werden und erfordern keine weitere Konfiguration. Zur Nutzung im Gesamtsystem wurden LED-Module entwickelt, bei denen jede LED über eine Transistorschaltung angesteuert wird, programmtechnisch ergibt sich daraus jedoch keine Änderung.

\section{Schrittmotor}
Die Schrittmotoren werden jeweils über einen Motortreiber und vier GPIO-Pins des Raspberry Pi angesteuert. Energie erhält der Motor durch ein an den Treiber angeschlossenes Netzteil mit 12V Gleichstrom. Durch paarweise Alternieren der Bits an den vier GPIO-Pins wird der Schrittmotor um jeweils einen Schritt bewegt. Es ergibt sich das Bitmuster in Tabelle \ref{wrap-tab:muster}.
\begin{wraptable}{r}{5cm}
	\begin{center}
		\begin{tabular}{|c|}
			\hline
			1010\\
			\hline
			1001\\
			\hline
			0101\\
			\hline
			0110\\
			\hline
		\end{tabular}
	\end{center}
	\caption{Bitmuster}\label{wrap-tab:muster}
\end{wraptable}

Der Schrittmotor kann nun bewegt werden, indem die Bitmuster nacheinander, unterbrochen durch eine Wartezeit, an den Schrittmotor geschickt werden. 
Die Geschwindigkeit des Schrittmotors kann durch die Wartezeit, die Richtung durch die Reihenfolge der Signale geändert werden.

Zusätzlich ist zu beachten, dass die Belegung mehrerer Pins des Raspberry Pi abwärts erfolgt, d.h. \texttt{RPiDigitalOut(19, 4)} belegt die Pins \texttt{16, 17, 18, 19}. 
Das Speichern von \texttt{BIT(4)} Variablen in einem Array funktioniert in OpenPEARL aktuell noch nicht; aufgrund der wenigen Zustände wurde auf einen Workaround verzichtet.

In diesem Schritt wurde als Modul eine Prozedur entwickelt, die einen spezifizierten Motor um eine spezifizierte Schrittzahl in spezifizierte Richtung bewegt. Mittels zweier Tasks können so auch beide Motoren parallel angesteuert werden.

Auf quantitave Dokumentation von Drehmoment und maximaler Geschwindigkeit wurde verzichtet, es wurde stattdessen lediglich die qualitative Eignung zur Nutzung im Gesamtsystem festgestellt.


\section{Lichtrechen}
Der Lichtrechen ist essenzieller Bestandteil des Autos und ist dafür zuständig die Position des Autos relativ zum vorgegebenen Pfad zu ermitteln. Der Lichtrechen QTR-8A der Marke Pololu besteht aus acht Reflexionssensoren, die mit Infrarot-LEDs die Reflexivität des Untergrundes auf einer Linie von ca. 7cm ermitteln. In unseren Anwendungsfall soll der Sensor ein schwatzes Klebeband auf einem hellen Untergrund erkennen und diese Information zur Steuerung des Autos weiterleiten. 

Die Verbindung zum System geschieht über acht separate Datenpins, die jeweils konstant ein einzelnes Bit, und damit die Auswertung eines einzelnen Sensors, binär repräsentieren. Des Weiteren gibt es noch einen Abschluss für die Stromzufuhr für 5 und 3.3V, einen Pin für die Erdung und einen Pin, mit dem die Infrarot-LED’s an oder abgeschaltet werden können. In unserem Aufbau nutzen wir die 5V Stromversorgung und lassen die LED-Steuerungspin unangeschlossen, in welchem Fall die Sensoren dauerhaft eingeschaltet bleiben. 

Im Anschlussplan ist vorgesehen, dass der Lichtrechen über die Datenpins 6 -13 am Raspberry Pi ausgelesen wird.

Bei der Auswertung der Datenpins werden den Pins von rechts nach links Gewichtungen von 4 bis -4 gegeben und jeder Pin der wenig Reflexion misst (Bit ist null) wird seiner Wertung nach aufsummiert. Diese Summe wird dann durch die Anzahl des Summanden geteilt und ergibt einen Durchschnittswert, der für die Position des Leitstreifens relativ zur Mitte des Sensors steht. Dieses Verfahren schwächt auch fehlerhafte Werte einzelner Sensoren ab. Lücken im Bitmuster verfälschen zwar den Ausgabewert, jedoch hängt mit diesem Verfahren das Gesamtergebnis nie von einem einzelnen Sensor ab und der Fehler wird somit abgeschwächt. 

Im Versuchsaufbau ist der Lichtrechen am vordern Ende des Autos auf der Unterseite der Bodenplatte angebracht, wo er knapp über dem Boden befestigt ist, um die Fehlerrate zu minimieren.


\section{Farbsensor}
Der eingesetzte Farbsensor \emph{TCS 34725} der Marke Adafruit besitzt einen Sensor, der sowohl einen RGB-Farbwert als auch die gesamt Helligkeit ermitteln lässt. Um die Farben bestmöglichst zu erkennen, besitzt der Sensor eine sehr helle LED mit weißem Licht. In unserem Projekt soll der Farbsensor farbige Signalpunkte auf der Strecke erkennen, anhand derer das Auto beispielsweise das Ende der Strecke erkennen soll. 

Die Datenverbindung zum Farbsensor stellt ein I2C-Bus her. Zu den üblichen zwei Datenverbindungen des I2C besitzt der Sensor ebenfalls einen Stromverbindung für 5 und 3.3V, eine Erdung und einen Pin zur Steuerung der LED.

Die Datenpins 2 und 3 am Raspberry Pi sind standardmäßig für diesen Bus vorkonfiguriert. Die Unterstützung von I2C auf dem Raspberry Pi muss aber erst konfiguriert werden. Dazu muss der Befehl \texttt{raspi-config} ausgeführt werden und dort unter “Advanced Options“ Die Unterstützung in Kernel eingeschaltet werden. Zusätzlich musste in der Datei \emph{/boot/config.txt} noch die Zeile \texttt{dtparam=i2c1=on} angefügt werden. (Dies erfolgte nach der detaillierten Anleitung in \footnote{https://developer-blog.net/raspberry-pi-i2c-aktivieren/, zuletzt abgerufen am: 20.02.2018}.)

Um den Farbsensor in Pearl benutzen zu können, musste dafür noch ein Treiben in C++ geschrieben werden. Dieser umfasst die Dateien \emph{TCS34725.c}, \emph{TCS34725.h} und \emph{TCS34725.xml}. Dieser Treiber wird in eine Headerfile von OpenPEARL integriert und stellt alle Methoden für die Nutzung des Sensors als \texttt{DATION} in OpenPEARL zur Verfügung. Dabei definiert der Treiber die Schnittstellen dationOpen zum initialisieren des Sensors und dationClose, um den Sensor zu schließen. Wenn der Sensor mit dationOpen geöffnet wurde, kann mit dationRead ein Datensatz vom Sensor ausgelesen werden. 

Bei der Auswertung der Rohdaten vom Sensor wollten wir die Daten in Standard RGB Werte mit einer Spanne von 0 bis 255 konvertieren, doch durch die Ungenauigkeit und große Differenz der Farbwerte war es einfacher und zuverlässiger mit den Rohdaten zu arbeiten. Rot und Grün wurden jeweils mit großem Ausschlag vom Sensor gemessen, wohingegen Blau eher schwierig zu bestimmen war.

Im Versuchsaufbau ist der Farbsensor an der Spitze des Autos auf der Unterseite der Bodenplatte angebracht, wo er, wie der Lichtrechen, knapp über dem Boden befestigt ist, um die Fehlerrate zu minimieren.

\section{TCP/IP Socket}
Über das Socket werden die Daten des SensorCars auf einer Webseite angezeigt. Hierbei bekommt das Socket einen html request vom Browser, an den es dann den HTML-Code schickt. Es werden die Geschwindigkeit des rechten sowie des linken Motors, der Status der Blinker und des Farbsensors angezeigt. Es können außerdem Informationen zum Projekt und die Dokumentation aufgerufen werden. 

Die Webseite kann im gleichen Netz über \emph{<ipadresse>/index.html} aufgerufen werden. Damit die Daten immer aktuell sind, wird die Seite alle zwei Sekunden aktualisiert.

OpenPEARL hat von sich aus keine Möglichkeit ein Socket aufzusetzen. Es musste daher zuerst implementiert werden. Hierzu wurde eine C++ Datei mit dazugehöriger XML Datei geschrieben und dem System hinzugefügt. Die Dateien wurden im Makefile und im \texttt{PearlIncludes.h} registriert. 

Um ein Socket aufzusetzen muss man im Systemteil von OpenPEARL TcpIpServer + den Port über den man kommunizieren will angeben. Im Problemteil kann man dann über \texttt{DATION INOUT} deklarieren. Um nun eine Anfrage erhalten zu können muss man das Socket über \texttt{OPEN} öffnen. Die Anfrage kann man dann mit \texttt{GET} abfragen und bearbeiten. Mit \texttt{PUT} schickt man Daten zurück. Zum Schluss muss man das Socket mit \texttt{CLOSE}  wieder schließen. Wenn dies in einer Dauerschleife läuft kann jede Anfrage bearbeitet werden. 
