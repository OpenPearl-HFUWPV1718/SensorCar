\chapter{Projektverlauf}
	
	\section{Organisation}
	
	Das Projektteam besteht aus den fünf Mitgliedern Kevin Hertfelder, Stefan Kienzler, Patrick Kroner, Daniel Petrusic und Daniel Schlageter. Betreut wird das Projekt durch Herrn Prof. Dr. Rainer Müller. In wöchentlichen Projekttreffen wird er aktuelle Status festgehalten und die Aufgaben für die nächste Woche, gemäß angepasster Planung, festgelegt. Die Protokolle dieser Sitzungen können im \href{https://github.com/OpenPearl-HFUWPV1718/SensorCar}{Repository} eingesehen werden.\\
	
	
	\section{Struktur}
	Das Projekt gliedert sich, entsprechend des oben beschriebenen Ziels, in mehrere Teilprojekte:
	
	\begin{itemize}
		\item \textbf{Projektstart}\\
		Einführung in das Projekt und Festlegung der Organisation. Grundlegende Abstimmung über Ziel und Umfang.
		\item \textbf{Einführung OpenPEARL und Philosophenproblem}\\
		 Zur Einarbeitung in die Programmiersprache OpenPEARL implementiert das Projektteam jeweils das Philosophenproblem.
		\item \textbf{Einrichtung der Raspberry Pis}\\
		 Für die weitere Arbeit am Projekt werden zwei Rasperberry Pi 3 eingerichtet, um mit OpenPEARL und NFS zu arbeiten. Der Zugriff auf die Rechner soll mittels \emph{ssh} möglich sein.
		\item \textbf{Inbetriebnahme und Test der Hardwarekomponente}\\  Die einzelnen Komponenten (Sensoren und Aktoren) werden als Teilprojekte separat am Raspberry Pi in Betrieb genommen und getestet.
		\item \textbf{Entwurf und Druck des Modells}\\
		Das Modell für den mechanischen Aufbau wird entwickelt und mit dem 3D-Drucker ausgedruckt und getestet.
		\item \textbf{Detailplanung des Gesamtsystems}\\  Die genaue Funktionalität und der Hardwareaufbau des Gesamtsystems werden festgelegt und dokumentiert.
		\item \textbf{Zusammensetzung Gesamtsystems}\\
		Nachdem alle Komponenten erfolgreich in Betrieb genommmen und getestet wurden, wird das Gesamtsystem im Sinne des Ziels zusammengebaut. Daraufhin wird die Funktionalität des Gesamtsystems programmiert. Im Sinne eines inkrementell iterativen Vorgehens erfolgen Anforderungsanalyse, Implementierung und Tests bis das System die Anforderungen erfüllt.
		\item \textbf{Projektabschluss}\\
		Zum Abschluss des Projekts erfolgt die Fertigstellung der Dokumentation und Abnahme des Ergebnisses.
		\item \textbf{Projektvorstellung}\\
		Das Ergebnis des Projektes wird im Rahmen der Projektpräsentationen vorgestellt.
	\end{itemize}

	\section{Ablauf}
	Tabelle \ref{tab:zeitplan} detailliert die grobe Zeitplanung für den Projektablauf. Änderungen sind vorbehalten.\\
	
	\begin{table}[H]
		
		\begin{center}
			\renewcommand{\arraystretch}{1.2}
			\begin{tabular}{|p{5cm}|l|}
				\hline
				\textbf{Zeitraum} & \textbf{Teilprojekt / Aufgabe}\\
				\hline
				09.10.17 -- 16.10.17 & Projektstart\\
				\hline
				16.10.17 -- 23.10.17 & Einführung in OpenPEARL und Philosophenproblem\\
				\hline
				23.10.17 -- 30.10.17 & Einrichtung der Raspberry Pis\\
				\hline
				30.10.17 -- 11.12.17 & Inbetriebnahme und Test der Hardwarekomponenten\\
				\hline
				30.10.17 -- 18.12.17 & Entwurf und Druck des Modells\\
				\hline
				11.12.17 -- 15.01.18 & Zusammensetzung des Gesamtsystems\\
				\hline 
				15.01.18 -- 22.01.18 & Projektabschluss\\
				\hline
				26.01.18 & Projektpräsentation\\
				\hline
			\end{tabular}
		\end{center}
		\caption{Zeitplan}\label{tab:zeitplan}
	\end{table}
	