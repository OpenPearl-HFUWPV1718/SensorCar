\chapter{Fazit}

Hauptziel des Projektes war der Einsatz von OpenPEARL in einem größeren System mit meheren Sensoren und Aktoren, um den Zustand von OpenPEARL und die Eignung der Sensoren festzustellen.\\
Die Aussage, dass OpenPEARL weit genug entwickelt ist, um es in ersten Projekten einzusetzen, konnten wir bestätigen, tatsächlich ist es uns gelungen ein Modellauto zu entwickeln, das in den meisten Fällen autonom einer Linie folgen konnte. Auch haben wir einige Fehler im OpenPEARL Compiler gefunden und konnten so zur Weiterentwicklung bzw. Verbesserung des Compilers beitragen.\\
Bezüglich der Sensorik mussten wir in einigen Fällen feststellen, dass der Einsatz in einem Echtzeitsystem oftmals schwieriger ist als gedacht, zumal die Qualität der Daten nicht immer optimal ist. Grundsätzlich war OpenPEARL jedoch gut geeignet, um viele verschiedenartige Geräte anzusteuern und in einem System einheitlich zu verwenden.\\
Das entstandene SensorCar war bei der Projektpräsentation im Allgemeinen funktionsfähig, aufgrund fehlender Tests gab es allerdings auch noch einige Probleme.\\
Da das Projektziel, OpenPEARL einzusetzen und zu testen klar erreicht wurde, können wir von einem Erfolg des Projektes sprechen.
